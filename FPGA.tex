\documentclass[11pt]{article}
\usepackage{hyperref}
\usepackage{url}
%Gummi|065|=)
\title{\textbf{Low Cost FPGA}}
\author{Kalpesh Krishna\\
		Abhin Shah}
\date{}
\begin{document}

\maketitle

\section{Altera MAX V }

We verified the Altera Chip soldered to the PCB. We used the similarities with the Max V chip on the Krypton Board used in Digital Systems lab.
 We took a Krypton Board without the Daughter card and the Altera Max V chip. The daughter card provided a feedback to the main board. Hence we needed to short two ground pins on the main Krypton Board which were equivalent to this feedback so that we had a common ground. We used a basic VHDL code to turn on / off LED using a switch to test this chip.
 
 \section{FT2232H}
 We verified the FT2232H cicuit on the PCB using the \href{http://www.ftdichip.com/Support/Documents/DataSheets/ICs/DS_FT2232H.pdf}{Datasheet} . We first verified it using the Krypton Board and then using the Altera MAX V chip on the PCB. The FT2232H output (pin 49) is further used to feed the 1.8 Volts voltage rail. We used an external circuit comprising of a voltage regulator for 3.3 Volts.
 
 

\section{CDCE925-Clock Synthesizer and I2C}
We learnt the I2C protocol. Our aim was to program the CDCE925 chip of Texas instruments using I2C according to the guidelines in the \href{http://www.ti.com.cn/cn/lit/ds/symlink/cdce925.pdf}{datasheet}. 
\\First of all we established I2C connection between two arduinos. We used Arduino Uno and Arduino Leonardo as Master and Slave respectively. We used Master Writer/Slave Receiver codes for the Master sending the data to slave and Master Reader/Slave Sender codes for the slave sending the data to the Master.Then we used I2C to program an RTC using Arduino. This was done for a basic understanding of the I2C protocol.
\\
Then we used the Wire library of Arduino to program CDCE925 chip. We set the output frequency to be 50 MHz.




\end{document}
